\documentclass[10pt,letterpaper]{article}
\usepackage[letterpaper,margin=1in]{geometry}
\usepackage{graphicx}
\usepackage[hyphens]{url}
\usepackage{hyperref}
\usepackage{listings}

\hypersetup{colorlinks=true,
breaklinks=true,
urlcolor=blue,
citecolor=NavyBlue}

\begin{document}
\lstset{language=Python, basicstyle=\small}

\title{Question \#1}
\author{Matt Chaney}

\maketitle

\begin{abstract}
Demonstrate that you know how to use "curl" well enough to 
correctly POST data to a form.  Show that the HTML response that
is returned is "correct".  That is, the server should take the
arguments you POSTed and build a response accordingly.  Save the
HTML response to a file and then view that file in a browser and
take a screen shot.
\end{abstract}

\section{Resources}
\begin{itemize}
\item \LaTeX: \url{http://www.electronics.oulu.fi/latex/examples/example_1}
\item \LaTeX: \url{http://scott.sherrillmix.com/blog/programmer/displaying-code-in-latex/}
\item curl: \url{http://curl.haxx.se/docs/httpscripting.html#POST}
\end{itemize}

\section{Results}
Using a simple wiki server built from the Go language net/http package tutorial, found here: \url{https://golang.org/doc/articles/wiki/}
\begin{verbatim}
[mchaney@mchaney-d gowiki]$ curl -v -d "body=something something" localhost:8080/save/TestPage
* About to connect() to localhost port 8080 (#0)
*   Trying ::1...
* connected
* Connected to localhost (::1) port 8080 (#0)
> POST /save/TestPage HTTP/1.1
> User-Agent: curl/7.27.0
> Host: localhost:8080
> Accept: */*
> Content-Length: 24
> Content-Type: application/x-www-form-urlencoded
> 
* upload completely sent off: 24 out of 24 bytes
< HTTP/1.1 302 Found
< Location: /view/TestPage
< Date: Fri, 29 Aug 2014 13:32:01 GMT
< Content-Length: 0
< Content-Type: text/plain; charset=utf-8
< 
* Connection #0 to host localhost left intact
* Closing connection #0
\end{verbatim}

\end{document}