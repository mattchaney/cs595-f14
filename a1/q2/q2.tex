\documentclass[10pt,letterpaper]{article}
\usepackage[letterpaper,margin=0.75in]{geometry}
\usepackage{graphicx}
\usepackage[hyphens]{url}
\usepackage{hyperref}

\hypersetup{colorlinks=true,
breaklinks=true,
urlcolor=blue,
citecolor=NavyBlue}

\begin{document}

\title{Question \#2}
\author{Matt Chaney}

\maketitle

\begin{abstract}
\begin{verbatim}
Write a Python program that:
  1. takes one argument, like "Old Dominion" or "Virginia Tech"
  2. takes another argument specified in seconds (e.g., "60" for 
     one minute).
  3. takes a URI as a third argument: 
     http://sports.yahoo.com/college-football/scoreboard/
     or
     http://sports.yahoo.com/college-football/scoreboard/?week=2&conf=all
     or
     http://sports.yahoo.com/college-football/scoreboard/?week=1&conf=72
     etc.
  4. dereferences the URI, finds the game corresponding to the team
     argument, prints out the current score (e.g., "Old Dominion 27, 
     East Carolina 17), sleeps for the specified seconds, and then
     repeats (until control-C is hit).
\end{verbatim}
\end{abstract}

\section{Resources}
\begin{itemize}
\item Requests: `\url{http://docs.python-requests.org/en/latest/}
\item BeautifulSoup: \url{http://www.crummy.com/software/BeautifulSoup/bs4/doc/}
\end{itemize}

\section{Results}
\begin{verbatim}
[mchaney@mchaney-d q2]$ python getscore.py "Texas A&M" 10 \
> "http://sports.yahoo.com/college-football/scoreboard/"
Texas A&M 52, South Carolina 28
Texas A&M 52, South Carolina 28
\end{verbatim}

\end{document}