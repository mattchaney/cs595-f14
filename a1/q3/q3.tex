\section{Question 3}

\subsection{Question}
Consider the "bow-tie" graph in the Broder et al. paper (fig 9): \url{http://www9.org/w9cdrom/160/160.html}\\\\
Now consider the following graph:
\begin{verbatim}
    A --> B
    B --> C
    C --> D
    C --> A
    C --> G
    E --> F
    G --> C
    G --> H
    I --> H
    I --> J
    I --> K
    J --> D 
    L --> D
    M --> A
    M --> N
    N --> D
\end{verbatim}

\subsection{Resources}
\begin{itemize}
\item Graph Structure in the web: \url{http://www9.org/w9cdrom/160/160.html}
\item Stanford, The web graph: \url{http://nlp.stanford.edu/IR-book/html/htmledition/the-web-graph-1.html}
\item Notes from the class:\\
SCC: Strongly Connected Component - all contained nodes are interconnected\\
IN: Connects into SCC, but not out from SCC\\
OUT: Connects out from SCC, but not in to SCC\\
Tendril: In or out excluding all SCC\\
Tube: IN->OUT or OUT->IN connection\\
Disconnected: Not connected to other sites\\
\end{itemize}

\subsection{Answer}
For the above graph, give the values for:\\\\
    IN: A, B, C, G\\
    SCC: M\\
    OUT: D, H\\
    Tendrils: L, K, I, J\\
    Tubes: N\\
    Disconnected: E, F\\