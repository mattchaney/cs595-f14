\section{Question 1}

\subsection{Question}
Write a Python program that extracts 1000 unique links from
Twitter.  You might want to take a look at:\\

http://thomassileo.com/blog/2013/01/25/using-twitter-rest-api-v1-dot-1-with-python/\\

But there are many other similar resources available on the web.  Note
that only Twitter API 1.1 is currently available; version 1 code will
no longer work.\\

Also note that you need to verify that the final target URI (i.e., the
one that responds with a 200) is unique.  You could have different
shortened URIs for www.cnn.com.  For example, \\

http://cnn.it/1cTNZ3V
http://t.co/BiYdsGotTd\\

Both ultimately redirect to cnn.com, so they count as only 1 unique URI.
Also note the second URI redirects twice -- don't stop at the first
redirect.\\

You might want to use the search feature to find URIs, or you can
pull them from the feed of someone famous (e.g., Tim O'Reilly).\\

Hold on to this collection -- we'll use it later throughout the semester.\\

\subsection{Resources}
\begin{itemize}
\item Getting Started with Twitter API: \url{http://thomassileo.com/blog/2013/01/25/using-twitter-rest-api-v1-dot-1-with-python/}
\item Twitter Search API: \url{https://dev.twitter.com/rest/public/search}
\item Twitter API - Get / Search Tweets: \url{https://dev.twitter.com/rest/reference/get/search/tweets}
\end{itemize}

\subsection{Answer}

Using the python module requests made this task a breeze as well as the initial code provided by Thomas Sileo's blog post.

\lstinputlisting[language=Python]{q1/urifinder.py}

The script was run multiple times to get the desired 1000 unique URIs. It would end prematurely at times, so the data set was initialized with the data of the previous run and then passed on to the {\tt find\_uris} function to preserve work performed.
